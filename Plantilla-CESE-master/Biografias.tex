
% Chapter 1

\chapter*{Biografias} % Main chapter title

%\label{Chapter1} % For referencing the chapter elsewhere, use \ref{Chapter1} 
%\label{IntroGeneral}


\textbf{\huge{Biografías 2}}

$\centerdot$ \textbf{Roberto Suárez Antola}; es Licenciado en Física, Magíster en Biofísica y Doctor en Ciencias Biológicas por la Universidad de la República de Uruguay (UdelaR).  Además posee formación en medicina adquirida en la Facultad de Medicina de UdelaR. Se especializó en ingeniería nuclear, a nivel de posgrado, en la Facultad de Ingeniería de la Universidad de Buenos Aires. Efectuó estudios de profundización en algunos temas de ciencias fisicomatemáticas e ingeniería en Alemania, Austria, España, Francia e Inglaterra.
Fue Profesor Titular y Director de Carrera en UdelaR y en la Universidad Católica del Uruguay. Dirigió los laboratorios de la Comisión Nacional de Energía Atómica y de la Dirección Nacional de Tecnología Nuclear del Ministerio de Industria, Energía y Minería (MIEM). Actualmente es Asesor en el MIEM de Uruguay. Compartió un premio de la Academia Nacional de Medicina, un premio Génesis del MIEM y un premio de la Academia Nacional de Ingeniería de Uruguay. Es autor de un libro sobre energía nuclear y un libro sobre teoría de la relatividad. Publicó artículos de investigación y capítulos de libro en temas de ciencias físico matemáticas, ingeniería y ciencias biomédicas.


$\centerdot$ \textbf{José E. Ruzzante}; es Licenciado en Física, por la Facultad de Ciencias Exactas y Naturales, de la Universidad de Buenos Aires, y recibido de Doctor en Física en la Universidad Nacional de La Plata, Argentina. Se especializó en Emisión Acústica en el CISE, Milán - Italia. Fue Director Científico del “International Centre for Earth Sciences” (ICES) que pertenece a la Comisión Nacional de Energía Atómica (CNEA) y a la Universidad Nacional de Cuyo (UNC), Mendoza – Argentina. Participó, junto a otros dos autores, en la escritura del libro "Ultrasonido y Emisión Acústica para Ingenieros y Estudiantes de Ingeniería", accesible desde: \url{https://www.researchgate.net/publication/341655775_"Ultrasonido-y-Emision Acustica-para-Ingenieros-y-Estudiantes-de-Ingenieria"}. y publicó el libro "Ondas elásticas en sólidos" ISBN 978-987-86-6502-3.
 Actualmente es:  Miembro Fundador del Grupo Latinoamericano de Emisión Acústica (GLEA);  Profesor Consulto de la Universidad Tecnológica Nacional (UTN);  Instructor ASME (American Society of Mechanical Engineers); Profesor Titular en la Universidad Nacional de Tres de Febrero (UNTREF) en la carrera de Ingeniería en Sonido;  Responsable del Grupo de Investigación en Acústica Subacuática (GIAS) de la misma Universidad.

$\centerdot$ \textbf{Pablo J.C. Alonso Castillo}; es Licenciado en Física, por la Facultad de Ciencias Exactas y Naturales, Ingeniero Electricista y Especialista en Sistemas Embebidos por la Facultad de Ingeniería, todas de la Universidad de Buenos Aires. Es Profesor Universitario  por la Facultad de Psicología y Pedagogía de la Universidad del Museo Social Argentino. Realizó estudios en los EE.UU., Trabajó como Ingeniero de Campo y Desarrollador de firmware y software para aplicaciones nucleares. Trabaja desde el año 2003 para la Comisión Nacional de Energía Atómica (CNEA), Lideró los desarrolos de diversos proyectos en el área de olfatometría electrónica y monitoreo remoto. Es el responsable del Laboratorio de Espectrometría de Movilidad Iónica (LEMI). Desde el año 2014 es el Secretario Académico del Instituto de Formación Técnica Superior N°14 del GCBA.
\chapter{Interacción de Canje} % Main chapter title

\begin{center}

Fue de casa en casa arrastrando dos lingotes metálicos, y todo el mundo se
espantó al ver que los calderos, las pailas, las tenazas y los anafes se caían de su sitio, y las maderas crujían por la desesperación de los clavos y los tornillos tratando de desenclavarse, y aun los objetos perdidos desde hacía mucho tiempo aparecían por donde más se les había buscado, y se arrastraban en desbandada turbulenta detrás de los fierros mágicos de Melquíades. "Las cosas tienen vida propia -pregonaba el gitano con áspero acento-, todo es cuestión de despertarles el ánima".

\hspace{3.6cm} Cien años de soledad\\
\hspace{4.6cm} Garcia Márquez

\end{center}


\section{Interacción de canje}


La interacción de canje fue descubierta independientemente por Heisenberg y Dirac en 1926 y está íntimamente relacionada con el del principio de exclusión de Pauli, 1925. Surge de forma natural al considerar la indistinguibilidad de algunas partículas. En mecánica clásica las partículas son distinguibles y se describen con la estadística de Maxwell Boltzmann en la mecánica cuántica no existe un procedimiento físico para distinguirlas o decir si una partícula observada en un instante es la misma que otra observada en un instante posterior. Esta circunstancia hace que el tratamiento cuántico adecuado de las partículas idénticas requiera la estadística Bose Einstein. El Ferromagnetismo es consecuencia del alineamiento de los espines de átomos adyacentess

Las fuerzas de canje dependen fundamentalmente de las distancias atómicas y no de posiciones atómicas la cristalinidad no es condición para el ferromagnetismo

La anisotropía magnética es la no homogeneidad de las propiedades magnéticas.

El ferromagnetismo no es particularidad de un tipo de estructura cristalina,
como vemos en la tabla. Lo que es característico de estos materiales es la presencia de capas $d$ o $f$ parcialmente llenas.